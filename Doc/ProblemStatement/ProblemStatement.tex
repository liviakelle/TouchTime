\documentclass{article}

\usepackage{tabularx}
\usepackage{booktabs}

\title{SE 3XA3: Problem Statement\\TouchTime}

\author{Team 13, Team Name
		\\ Matthew Po - pom
		\\ Livia Kelle - kellel
		\\ Evan Ansell - ansellea
}

\date{}

% \input{../Comments}

\begin{document}

\begin{table}[hp]
\caption{Revision History} \label{TblRevisionHistory}
\begin{tabularx}{\textwidth}{llX}
\toprule
\textbf{Date} & \textbf{Developer(s)} & \textbf{Change}\\
\midrule
September 24, 2017 & Matthew Po & Initial draft of Problem Statement\\
September 25, 2017 & Evan Ansell & Added to initial draft\\
Date3 & Name(s) & Description of changes\\
... & ... & ...\\
\bottomrule
\end{tabularx}
\end{table}

\newpage

\maketitle

\section{High Level Problem Summary}
Project Name: 
\textbf{TouchTime}
\vspace{0.3cm}

Hardware should detect when the hour hand and minute hand have been touched and respond with the appropriate number of vibrations. Software should indicate what time the hour and minute hand are currently on. Hardware should vibrate a particular pattern to indicate the time that the watch is currently on. The vibrations should be distinguishable and easy to understand. The patterns of the vibrations should be easy enough to translate into time. Designed to be useful for the vision-impaired and for people who want to tell the time surreptitiously.
\vspace{0.3cm}

\section{Importance of the Problem}

Nowadays the time is a very important thing to know. Whether it's for the time of a  meeting, the start of a class, or a bus schedule, people need to be aware of the time. Generally most people will have easy access to looking at the time since they carry a phone, a watch, or both. Unfortunately this isn't the case for all people, and those with a visual impairment may have difficulties with telling the time from a traditional watch or phone. While solutions currently exist for such people, they are often imprecise or tell the time by sound. Our solution of telling time by vibration will allow people to be able to tell the time in a more precise manner in situations which traditional solutions would fail such as a loud event or a situation in which it would be considered rude to be looking at your watch.

% \section{Detailed Problem Statement}
% Project Name: 
% \textbf{TouchTime}
% \vspace{0.3cm}

\subsection{}
\wss{comment}

\ds{comment}

\mj{comment}

\cm{comment}

\mh{comment}

\end{document}

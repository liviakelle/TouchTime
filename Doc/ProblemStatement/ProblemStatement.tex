\documentclass{article}

\usepackage{tabularx}
\usepackage{booktabs}


\title{SE 3XA3: Problem Statement\\TouchTime}

\author{Team 13, Team Name
		\\ Matthew Po - pom
		\\ Livia Kelle - kellel
		\\ Evan Ansell - ansellea
}

\date{}

\input{../Comments}

\begin{document}

\begin{table}[hp]
\caption{Revision History} \label{TblRevisionHistory}
\begin{tabularx}{\textwidth}{llX}
\toprule
\textbf{Date} & \textbf{Developer(s)} & \textbf{Change}\\
\midrule
September 24, 2017 & Matthew Po & Initial draft of Problem Statement\\
September 25, 2017 & Evan Ansell & Added to initial draft\\
September 25, 2017 & Livia Kelle & Added Project Context\\
\bottomrule
\end{tabularx}
\end{table}

\newpage

\maketitle

\section{High Level Problem Summary}
Project Name: 
\textbf{TouchTime}
\vspace{0.3cm}

Hardware should detect when the hour hand and minute hand have been touched and respond with the appropriate number of vibrations. Software should indicate what time the hour and minute hand are currently on. Hardware should vibrate a particular pattern to indicate the time that the watch is currently on. The vibrations should be distinguishable and easy to understand. The patterns of the vibrations should be easy enough to translate into time. Designed to be useful for the vision-impaired and for people who want to tell the time surreptitiously.
\vspace{0.3cm}

\section{Project Importance}
Project Name: 
\textbf{TouchTime}
\vspace{0.3cm}

Nowadays the time is a very important thing to know. Whether it's for the time of a  meeting, the start of a class, or a bus schedule, people need to be aware of the time. Generally most people will have easy access to looking at the time since they carry a phone, a watch, or both. Unfortunately this isn't the case for all people, and those with a visual impairment may have difficulties with telling the time from a traditional watch or phone. While solutions currently exist for such people, they are often imprecise or tell the time by sound. Our solution of telling time by vibration will allow people to be able to tell the time in a more precise manner in situations which traditional solutions would fail such as a loud event or a situation in which it would be considered rude to be looking at your watch.
\vspace{0.3cm}

\section{Project Context}
Project Name: 
\textbf{TouchTime}
\vspace{0.3cm}

The problem we are trying to solve involves three main stakeholders: the development team, the clients, and the users. The development team consists of the members of group 13, the clients are the TAs and professor of Software Engineering 3XA3, and the users are the people that will be using the application. The users will have access to either an Android Smartwatch or an Android Smartphone and are either people who are visually impaired or people who want to be able to tell the time without looking. With this special set of users in mind, the application interface should be simple to navigate. The application should require very little maintenance and should run on any Android Wear 2.0.0 or later and Android OS 7.1.1 or later. The application itself will be developed through the use of Android Studio and will be written using Java.


% \subsection{}
% \wss{comment}

% \ds{comment}

% \mj{comment}

% \cm{comment}

% \mh{comment}

\end{document}
